\section{Model}
\subsection{Basic Functions}
The netwok model relies on three types of entities (at present, as we model more complex functionality we might get to
other things):

\begin{itemize}
\item $node$: These are network elements including endhosts, switches (when modelled), middleboxes, etc. In the rest of
the document identifiers $e, e_1, e_2, \ldots$ are assumed to be $node$s.
\item $address$: Addresses assigned to nodes. In the rest of the document identifiers $a, b, a_1, a_2\ldots$ are assumed
to be $address$es
\item $packet$: Packets. We currently model them as tuples with the following fields:
\begin{itemize}
\item $src$: $src\in address$.
\item $dest$: $dest\in address$.
\item $origin$: $origin\in node$. This is a pseudofield in the packet tracking its true origin (which is important for
answering questions about isolation).
\end{itemize}
For the rest of the document identifiers $p, p_1,\ldots$ represent packets.
\end{itemize}

We also create a few basic functions to model relationships between these types and also network functionality. These
are:
\begin{itemize}
\item $hostHasAddr: e, a \rightarrow boolean$: True if $node$ $e$ is assigned address $a$.
\item $addrToHost: a \rightarrow e$: Returns $node$ $e$ associated with address $a$. (Note this implies that the address
to node mapping is onto).
\item $send: e_1, e_2, p \rightarrow boolean$ True if $e_1$ sent $e_2$ packet $p$.
\item $recv: e_1, e_2, p \rightarrow boolean$ True if $e_2$ received packet $p$ from $e_1$.
\end{itemize}

\textbf{Note:} We do not currently model time just a set of packets. For entities involved in learning (such as stateful
firewall) this complicates the generation of test cases/counter examples (since these might depend on the order of
packets sent). I think it is fairly simple to add time, I just need to change a substantial portion of the model, which
I didn't want to do until I had at least all of the basic bits working.

Given these functions and types we layer them on to produce the test network. We go bottom up starting with the network:

\subsection{Network Layer}
\subsubsection{Basic Conditions}
These are basic conditions we assume for the entire network:
\begin{itemize}
\item $\forall e, a:\ hostHasAddr(e, a)\iff addrToHost(a) = e$: Addresses are symmetric
\item $\forall e_1, e_2, p:\ recv(e_1, e_2, p) \iff send(e_1, e_2, p)$: No packet loss
\item $\forall e_1, e_2, p:\ recv(e_1, e_2, p) \Rightarrow \exists e_3:\ send(addrToHost(p.src), e_3, p)$: Network doesn't
invent packets, packet source is correct.
\item Don't consider loopback packets, we have really no control over them.
\begin{align*}
\forall e_1, e_2, p:& send(e_1, e_2, p) \Rightarrow e_1 \neq e_2\\
\forall e_1, e_2, p:& recv(e_1, e_2. p) \Rightarrow e_1 \neq e_2\\
\end{align*}
\end{itemize}

\subsubsection{Adjacency Conditions}
These are used to impose network topology (i.e. deal with cases where two nodes aren't connected and cannot send
messages). A set of these conditions are imposed for each node in the network.
In this particular case $e$ is the node $adj \subset node$ is the set of nodes adjacent to $e$.
\begin{align*}
\forall e_i, p:& send(e, e_i, p) \Rightarrow e_i \in adj\\
\forall e_1, p:& recv(e_i, e, p) \Rightarrow e_i \in adj
\end{align*}

\subsubsection{Routing Tables}
These are essentially the same as routing tables in real networks. We can populate these using, for instance, the output
of HSA (since we can extend these conditions to match on arbitrary packet criterion). Currently they are expressed in
terms of destination:

For now a table $T \subset address\times node = \left\{ (a, e) \right\}$ is a set of $address$-$node$ tuples indicating what node gets the packet
next. Given such a table $T$ for a node $e$ we impose the following condition:
\begin{align*}
\forall t\in T:& \forall e_i, p:\ send(e, e_i, p) \land p.dest = t.a \Rightarrow e_i = t.e
\end{align*}

\subsubsection{Correctly Send}
This is just a condition that we add to a lot of the middleboxes (it is not assumed by default). Below we mark when this
is used. It basically says that a node $e$ will only send a packet $p$ if $p$'s destination is not $e$.
\begin{align}\label{eq:sanesend}
\forall e_1, p:& send(e, e_1, p) \Rightarrow \neg hostHasAddr(e, p.dest)
\end{align}

\subsection{End Hosts}
End hosts are just nodes with a couple of extra constraints. For endhost $e$:
\begin{itemize}
\item $\forall e_i, p:\ send(e, e_i, p) \Rightarrow hostHasAddr(e, p.src)$: An end host does not forward packets on anyone
elses behalf. Note that this prevents host spoofing. One could obviously turn this off, but...
\item $\forall e_i, p:\ send(e, e_i, p) \Rightarrow p.origin = e$ Track origin correctly.
\end{itemize}

\subsection{Stateless Firewall}
Stateless firewalls just implement ACLs based on packet source and destination. ACLs are specified as a table $A \subset
address\times address = \left\{(a, b)\right\}$ which specifies the set of addresses that are denied (it is equally easy
to build one based on allowed addresses). Given a firewall $f$ and a table $A$, the firewall model says:

\begin{itemize}
\item The correct sending condition from equation~\ref{eq:sanesend}.
\item $\forall e_1, p:\ send(f, e_1, p) \Rightarrow \exists e_2 recv(e_2, f, p)$: The firewall does not invent packets.
\item $\forall e, p:\ send(f, e, p) \Rightarrow (p.src, p.dest) \not \in A \land (p.dest, p.src) \not \in A$ Sending the packet is not disallowed by an
ACL.
\end{itemize}

\subsection{Stateful Firewall}
Stateful firewally can cache previous decisions. In particular this means we don't need to have ACLs that are symmetric
(one side can punch a hole through the firewall). Cached rules are modeled as a function: $cached: a_1, a_2 \rightarrow
boolean$ which is true when packets with either source $a_1$ and destination $a_2$ or source $a_2$ and destination $a_1$
are allowed through the firewall. The logical model for a stateful firewall $f$ with ACL table $A$ is:

\begin{itemize}
\item The correct sending condition from equation~\ref{eq:sanesend}.
\item $\forall a, b:\ cached(a, b) \iff \exists e, p:\ recv(e, f, p) \land p.src = a \land p.dest = b\land (a, b) \not
\in A$: Cache based on packets that are received.
\item $\forall e, p:\ send(f, e, p) \Rightarrow cached(p.src, p.dest) \lor cached(p.dest, p.src)$. Send if cached.
\end{itemize}

\subsection{Web Proxy}
Web proxies modify packet headers (and were our original source of violations). Given a web proxy $w$ we model it as
follows:
\begin{itemize}
\item $\forall e, p:\ send(w, e, p) \implies hostHasAddr(w, p.src)$ Send all packets so source address belongs to the
proxy, thus allowing caching.
\item $\forall e_1, p_1:\ send(w, e, p) \implies \exists e_2, p_2:\ recv(e_2, w, p_2) \land p_2.origin = p.origin \land
p_2.dest = p.dest \land hostHasAddr(p_2.origin, p_2.src)$ rules for what packets a webproxy sends. These are actually
somewhat incomplete (we are not accounting for responses here). The reason is that currently I model all endhosts as
endhosts instead of as servers and client, which would make this somewhat more reasonable.
\end{itemize}
